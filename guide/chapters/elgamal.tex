\ProvidesFile{elgamal.tex}

\section{Эллиптический Эль-Гамаль}
Общие данные: простое число $p$, эллиптическая кривая $\E(\F_p)$, точка на этой кривой $P\in\E(\F_p)$, её порядок $q$, отображение сообщения на эллиптическую кривую и обратное слева к нему отображение.

  Допустим Боб хочет отправить сообщение $m$ Алисе. Тогда:
  \begin{enumerate}
    \item Алиса генерирует секретный ключ $n_\alpha\colon 1\leqslant n_\alpha<q$, публичный ключ: $A:=n_\alpha P$. Отправляет публичный ключ Бобу.
    \item Боб обратимо отображает сообщение $m$ на эллиптическую кривую: $M\in\E(\F_p)$, генерирует одноразовый сектретный ключ $k\colon 1\leqslant k<q$, вычисляет $C_1:=kP,\q C_2:=kA+M$ и отправляет пару $(C_1,C_2)$ Алисе.
    \item Алиса вычисляет исходную точку по формуле:
    \[C_2 - n_\alpha C_1 =\]\[= kA+M - n_\alpha k P = n_\alpha k P + M - n_\alpha k P = M\]
    и отображает её обратно в $m$ 
  \end{enumerate}
  
  Отображать сообщение, представленное в виде двоичного числа, на эллиптическую кривую можно следующим образом:
  \begin{enumerate}
    \item Пусть наше поле $\F_p$, и $p$ - это достаточно большое число. Обозначим за $q$ - длину числа $p$ в битовом представлении.
    \item Выберем и зафиксируем число $l\in (0, \frac{q}{2})$, и будем в первые $l$ бит числа записывать наше сообщение $m$ в двоичном представлении.
    \item Заполним оставшиеся биты случайно. Тогда высока вероятность, что полученное число - это координата $x$ какой-то точки на эллиптической кривой. Если нет, то повторяем этот шаг, пока не получим точку на кривой.
    \item Когда Алиса посчитала точку $M$, она берёт координату $x$ этой точки. Тогда первые $l$ бит значения этой координаты и будут сообщением $m$. 
  \end{enumerate}

  Заметим, что отображать сообщения на кривую и обратно не является удобным. В вариации Эль-Гамаля с хешированием, выбирается хеш-функция $H\colon \E(\F_p)\to \left\{0,1\right\}^n$ и $(C_1,c_2) := (kP, m \oplus H(kA))$, т.е. сообщение в двоичном виде длины $\leqslant n$ XORится с хешем от точки $kA$. Тогда Алиса получает исходное сообщение через $m = c_2 \oplus H(n_\alpha C_1)$.

Реализуем данный функционал через класс:
\subsection{Каркас}
Запрашиваем нужные общие данные:
\begin{cppcode}
class ElGamal {
    using Curve = EllipticCurve;
    using Point = EllipticCurvePoint;

public:
    ElGamal(const Curve& curve, const Point& generator, const uint& generator_order);

private:

    Curve m_curve;
    Point m_generator;
    uint m_generator_order;
};
\end{cppcode}
\subsection{Методы}
Нам нужны методы
\begin{enumerate}
  \item Генерации ключей.

  Для этого нужнен генератор рандомных чисел. Обычный мерсен не подойдёт, так как он не предназначен для криптографии. Используем \href{https://github.com/Duthomhas/CSPRNG}{библиотеку CSPRNG} для этих целей. Она использует внутренние генераторы операционной системы. Напишем генератор рандомных uint:
  \begin{cppcode}
#include "csprng/csprng.hpp"

constexpr size_t c_size = 512 / sizeof(uint32_t);

uint generate_random_uint() {
    duthomhas::csprng rng;
    std::seed_seq sseq {228};
    rng.seed(sseq);
    uint result = 0;
    uint32_t x = rng(uint32_t());
    std::vector<uint32_t> values = rng(std::vector<uint32_t>(c_size));

    for (size_t i = 0; i < c_size; ++i) {
        result <<= 32;
        result += values[i];
    }

    return result;
}

uint generate_random_uint_modulo(const uint& modulus) {
    return generate_random_uint() % modulus;
}

uint generate_random_non_zero_uint_modulo(const uint& modulus) {
    return generate_random_uint_modulo(modulus - 1) + 1;
}
  \end{cppcode}

  Тогда ключи получаем через:
  \begin{cppcode}
struct Keys {
    uint private_key;
    Point public_key;
};

Keys ElGamal::generate_keys() const {
    uint private_key = generate_random_non_zero_uint_modulo(m_generator_order);
    Point public_key = private_key * m_generator;
    return Keys {.private_key = private_key, .public_key = public_key};
}
  \end{cppcode}
  \item Шифрование сообщения. Есть два варианта - нам предоставили уже готовое сообщение $M\in\E(\F_p)$, либо надо самим отобразить $m$ на кривую. Напишем отображение uint на кривую, через вероятностный алгоритм:
  \begin{cppcode}
size_t actual_bit_size(uint value) {
    size_t result = 0;

    while (value != 0) {
        ++result;
        value >>= 1;
    }

    return result;
}

static std::map<uint, uint> p_zero_mask;
static constexpr uint c_full_bits = uint(0) - 1;

ElGamal::Point ElGamal::map_to_curve(const uint& message) const {
    const Field& F = m_curve.get_field();
    const uint& p = F.modulus();

    if (!p_zero_mask.contains(p)) {
        const size_t l = actual_bit_size(p) >> 1;
        uint zero_mask = (c_full_bits >> l) << l;
        p_zero_mask.insert({p, zero_mask});
    }

    const uint& zero_mask = p_zero_mask.at(p);

    for (;;) {
        uint x = random::generate_random_uint_modulo(p);
        x &= zero_mask;
        x |= message ^ (message & zero_mask);
        auto opt = m_curve.point_with_x_equal_to(F.element(x));

        if (opt.has_value()) {
            return opt.value();
        }
    }
}
  \end{cppcode}
  Так как нам нужно быстро записывать сообщение в половину длины простого числа, то нам нужна нулевая маска для этой половины. Заводим статическую мапу, которая будет хранить предподсчитанные маски, и создаём новые маски, если такого простого числа p не встречалось за время выполнения.

  Отображение обратно с точки на uint происходит аналогично:
  \begin{cppcode}
uint ElGamal::map_to_uint(const Point& message) const {
    const uint& p = m_curve.get_field().modulus();

    if (!p_zero_mask.contains(p)) {
        const size_t l = actual_bit_size(p) >> 1;
        uint zero_mask = (c_full_bits >> l) << l;
        p_zero_mask.insert({p, zero_mask});
    }

    const uint& zero_mask = p_zero_mask.at(p);
    uint x = message.get_x().value();
    x ^= (x & zero_mask);
    return x;
}
  \end{cppcode}
  Заметим, что мы обрезаем сообщение, если количество бит в нём больше, чем половина бит от простого модуля. С высокой долей вероятности, этот алгоритм будет работать относительно быстро и при большей заполненности бит простого модуля, поэтому вы можете поэксперементировать над длиной нулевой маски.

  Имея оба отображения, можем имплементировать алгоритм шифрования, описанный ранее:
  \begin{cppcode}
struct EncryptedMessage {
    Point generator_degree;
    Point message_with_salt;
};

EncryptedMessage encrypt(const Point& message, const Point& public_key) const {
    const uint k = generate_random_non_zero_uint_modulo(m_generator_order);
    const Point generator_degree = k * m_generator;
    const Point message_with_salt = message + k * public_key;
    return {.generator_degree = generator_degree, .message_with_salt = message_with_salt};
}

EncryptedMessage encrypt(const uint& message, const Point& public_key) const {
    Point point_message = map_to_curve(message);
    return encrypt(point_message, public_key);
}
  \end{cppcode}
  \item Дешифрование. По алгоритму, описанному ранее:
  \begin{cppcode}
Point decrypt_to_point(const EncryptedMessage& encrypted_message, const uint& private_key) const {
    return encrypted_message.message_with_salt - private_key * encrypted_message.generator_degree;
}

uint decrypt_to_uint(const EncryptedMessage& encrypted_message, const uint& private_key) const {
    const Point M = encrypted_message.message_with_salt
                    - private_key * encrypted_message.generator_degree;
    return map_to_uint(M);
}

  \end{cppcode}
\end{enumerate}

Итого сигнатура класса:
\begin{cppcode}
class ElGamal {
    using Curve = EllipticCurve;
    using Point = EllipticCurvePoint;

public:
    struct Keys {
        uint private_key;
        Point public_key;
    };

    struct EncryptedMessage {
        Point generator_degree;
        Point message_with_salt;
    };

    ElGamal(const Curve& curve, const Point& generator, const uint& generator_order);

    Keys generate_keys() const;

    EncryptedMessage encrypt(const uint& message, const Point& public_key) const;
    EncryptedMessage encrypt(const Point& message, const Point& public_key) const;
    Point decrypt_to_point(const EncryptedMessage& encrypted_message, const uint& private_key) const;
    uint decrypt_to_uint(const EncryptedMessage& encrypted_message, const uint& private_key) const;

private:
    Point map_to_curve(const uint& message) const;
    uint map_to_uint(const Point& message) const;

    Curve m_curve;
    Point m_generator;
    uint m_generator_order;
};
\end{cppcode}

