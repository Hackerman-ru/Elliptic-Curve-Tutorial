\ProvidesFile{introduction.tex}

\begin{abstract}
  Работа является пошаговым руководством по реализации криптографии на эллиптических кривых. Реализованы объекты длинной арифметики, полей и эллиптических кривых. Изучены и имплементированы алгоритмы шифрования и дешифрования, электронной цифровой подписи, подсчёт количества точек на эллиптической кривой, быстрого умножения и деления длинных чисел. Протестированы объекты и алгоритмы по скорости, сравнивая с готовыми решениями. Руководство параллельно с имплементацией объясняет и рассказывает, что и зачем было реализовано.

  \textit{Ключевые слова: эллиптические кривые, шифрование и дешифрование, криптография, ECDSA, ECC, длинная арифметика, FFT, C++, конечные поля, оптимизация, Schoof's algorithm}
\end{abstract}

\section{Введение}
Современная криптография с нынешними вычислительными мощностями требует значительных ухищрений в шифровке сообщений, и шифрование с помощью эллиптических кривых - один из мощнейших инструментов. Но доступных и полных объяснений от начала до конца по шифрованию на них ничтожно мало, поэтому я решил сделать руководство для людей, которые хотят ознакомиться с данным видом криптографии.

\subsection{Условия игры}
 Если вы искали данное руководство, то скорее всего где-то слышали/читали об эллиптических кривых и о возможности криптографии на них, поэтому я рассчитываю на базовое понимание математики и алгоритмов.

Здесь не будет дотошного доказательства теорем или строгости в описании математических объектов --- в первую очередь акцент делается именно на имплементации (на языке C++). Данный язык был выбран в качестве общеизвестного языка среди программистов. Выберем C++20 для удобного использования шаблонов.

В данном руководстве мы будем стараться использовать как можно меньше готовых библиотек, чтобы не было огромных black box-ов в нашем коде. Это улучшит понимание и возможности алгоритмов.

\subsection{База}
Знаменитая формула
\[y^2 = x^3 + ax + b\]
обычно является самой первой, которую вы увидите при описании криптографии эллиптических кривых. Появляется несколько вопросов:
\begin{itemize}
  \item Что такое $x,y,a,b$? Где лежат данные числа?

  Данные числа являются элементами некоего поля $\F$, над которым построенна эллиптическая кривая, характеристики больше 3 (забьём на последние слова, так как мы будем работать с полями достаточно больших характеристик). Поле поддерживает все стандартные математические операции: сложение, вычитание, деление на ненулевой элемент, умножение, поэтому можно пока считать его $\R$.
  \item Что такое эллиптическая кривая?

  Это группа точек в $\F^2$, координаты которых удовлетворяют данному уравнению, и ещё точка бесконечности $\mathcal{O}$, которая является своеобразным нулём группы. Сложение в группе происходит по специальным формулам на координаты, которые будут рассмотрены позже. Умножение точки на натуральное число приравнивается к сложению точки с собой это число раз.
  \item Как это используют для шифрования?

  Обычно выбирается эллиптическая кривая $\E$ над неким полем $\F$, точка $P$ на ней и производится умножение точки на натуральное число $k$. Криптографическая стойкость достигается сложностью нахождения числа $k$ по точкам $P$ и $kP$. 
  \item Чем это лучше других методов шифрования?

  Тем, что данный способ шифрования можно реализовать так, что он будет выполнятся быстрее других алгоритмов при аналогичной задаче и данных. Также, для одинаковых показателей криптографической стойкости, криптография на эллиптических кривых требует ключей (чисел для шифрования) меньшей длины, чем другие алгоритмы.
\end{itemize}

\subsection{Постановка задачи}
Начитавшись статей на хабре, мы воодушевились и решили написать свою криптографию на эллиптических кривых. Сначала надо определить, какие объекты нам надо реализовать:
\begin{itemize}
  \item Нам надо реализовать эллиптическую кривую. Но эллиптическая кривая никто без поля, значит нам надо реализовать поле.
  \item Так как поля бывают бесконечными, а мы работаем на компьютере с числами, то ограничимся на простые поля $\F_p$, которые представим в виде вычетов по простому модулю $p$. Но этот простой модуль и числа в поле надо представить в виде целых чисел, а в криптографии обычно используются числа из более чем 200 битов. Целые числа такого размера не поддерживаются языком C++, поэтому нам надо реализовать класс целых чисел и длинную арифметику на них.
\end{itemize}

Итого 3 объекта: целые числа, поле, эллиптическая кривая. Приступим наконец к реализации!

