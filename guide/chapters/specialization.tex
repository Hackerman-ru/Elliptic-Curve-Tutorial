\ProvidesFile{specialization.tex}

\section{Специализация}
Остановимся на конкретной кривой и генераторе, а именно \href{https://neuromancer.sk/std/nist/P-256}{NIST P-256}. Будем писать однофайловое решение шифрования эллиптическим Эль-Гамалем. Начнём!

\subsection{Поле}
Так как нам нужно одно конкретное поле, то можно соединить классы FieldElement и uint\_t в одно целое, забыв о классе Field:
\begin{cppcode}
class F_256 {
    static constexpr size_t c_bits = 512;
    static constexpr size_t c_bits_in_byte = 8;
    static constexpr size_t c_digit_size = 32;
    static constexpr size_t c_digit_number = 16;
    static constexpr size_t c_half_digit_number = 8;

    using digits = std::array<uint32_t, c_digit_number>;
    digits m_digits = {};
}
\end{cppcode}
\subsubsection{Арифметика}
Так как наше простое число имеет 256 бит, то достаточно 512 бит для сдерживания переполнения. Для нормализации числа нам нужно определить само простое число p, но вот загвоздка --- оно то тоже 256 битное, а мы не можем определить статическое поле класса самим классом. Поэтому идём на хитрость: расписываем через constexpr только цифры простого числа:
\begin{cppcode}
static constexpr digits p_values = {4294967295U, 4294967295U, 4294967295U, 0U, 0U, 0U, 1U, 4294967295U};
\end{cppcode}
Тогда для прибавления и вычитания простого числа при нормализации числа определим приватные методы, которые не проверяют инвариант:
\begin{cppcode}
constexpr void add_p() {
    uint32_t carry = 0;

    for (size_t i = 0; i < c_digit_number; ++i) {
        uint32_t sum = carry + p_values[i];
        m_digits[i] += sum;
        carry = (m_digits[i] < sum) || (sum < carry);
    }
}

constexpr void subtract_p() {
    uint32_t remainder = 0;

    for (size_t i = 0; i < c_digit_number; ++i) {
        uint32_t prev = m_digits[i];
        uint32_t sum = p_values[i] + remainder;
        m_digits[i] -= sum;
        remainder = (m_digits[i] > prev) || (sum < remainder);
    }
}
\end{cppcode}
Специально повторили логику методов += и -=, чтобы не было бесконечного зацикливания.

Теперь модифицируем инкремент и декремент: так как мы считаем, что при инкременте число поддерживало инвариант, то максимум кем оно могло стать --- это самим простым модулем p, поэтому:
\begin{cppcode}
constexpr void increment() {
    for (size_t i = 0; i < c_digit_number; ++i) {
        m_digits[i] += 1;

        if (m_digits[i] != 0) {
            break;
        }
    }

    if (m_digits == p_values) {
        m_digits = {};
    }
}
\end{cppcode}
С декрементом немного посложней. Основное переполнение --- это декремент 0. Но мы знаем, чему число тогда станет равно: $p-1$, поэтому заранее определим его:
\begin{cppcode}
static constexpr digits max_mod_p = {4294967294U, 4294967295U, 4294967295U, 0U, 0U, 0U, 1U, 4294967295U};

constexpr void decrement() {
    if (*this == 0) {
        m_digits = max_mod_p;
        return;
    }

    for (size_t i = 0; i < c_digit_number; ++i) {
        uint32_t temp = m_digits[i];
        m_digits[i] -= 1;

        if (temp >= m_digits[i]) {
            break;
        }
    }
}
\end{cppcode}
Оператором отрицания будет как обычно вычитание из модуля или 0, если текущее значение 0:
\begin{cppcode}
constexpr F_256 operator-() const {
    if(!is_invertible) {
        return *this;
    }

    blocks result = p_values;
    uint32_t remainder = 0;

    for (size_t i = 0; i < c_digit_number; ++i) {
        uint32_t prev = result[i];
        uint32_t sum = m_digits[i] + remainder;
        result[i] -= sum;
        remainder = (result[i] > prev) || (sum < remainder);
    }

    return F_256(result);
}
\end{cppcode}
Теперь можно определить операции сложения и вычитания:
\begin{cppcode}
constexpr F_256& operator+=(const F_256& other) {
    uint32_t carry = 0;

    for (size_t i = 0; i < c_digit_number; ++i) {
        uint32_t sum = carry + other[i];
        m_digits[i] += sum;
        carry = (m_digits[i] < sum) || (sum < carry);
    }

    if (!is_valid()) {
        subtract_p();
    }

    return *this;
}

constexpr F_256& operator-=(const F_256& other) {
    uint32_t remainder = 0;

    for (size_t i = 0; i < c_digit_number; ++i) {
        uint32_t prev = m_digits[i];
        uint32_t sum = other[i] + remainder;
        m_digits[i] -= sum;
        remainder = (m_digits[i] > prev) || (sum < remainder);
    }

    if (!is_valid()) {
        add_p();
    }

    return *this;
}
\end{cppcode}
\subsubsection{Умножение}
Чтобы редуцировать число, которое получиться после умножения двух чисел из поля, нам надо научиться брать по модулю число $c\colon 0\leqslant c < p^2$. У P-256 есть специальный ритуал для этого:
\begin{cppcode}
constexpr void reduce() {
    F_256 s1({m_digits[0],
              m_digits[1],
              m_digits[2],
              m_digits[3],
              m_digits[4],
              m_digits[5],
              m_digits[6],
              m_digits[7]});
    F_256 s2({0, 0, 0, m_digits[11], m_digits[12], m_digits[13], m_digits[14], m_digits[15]});
    F_256 s3({0, 0, 0, m_digits[12], m_digits[13], m_digits[14], m_digits[15], 0});
    F_256 s4({m_digits[8], m_digits[9], m_digits[10], 0, 0, 0, m_digits[14], m_digits[15]});
    F_256 s5({m_digits[9],
              m_digits[10],
              m_digits[11],
              m_digits[13],
              m_digits[14],
              m_digits[15],
              m_digits[13],
              m_digits[8]});
    F_256 s6({m_digits[11], m_digits[12], m_digits[13], 0, 0, 0, m_digits[8], m_digits[10]});
    F_256 s7({m_digits[12], m_digits[13], m_digits[14], m_digits[15], 0, 0, m_digits[9], m_digits[11]});
    F_256 s8({m_digits[13],
              m_digits[14],
              m_digits[15],
              m_digits[8],
              m_digits[9],
              m_digits[10],
              0,
              m_digits[12]});
    F_256 s9({m_digits[14], m_digits[15], 0, m_digits[9], m_digits[10], m_digits[11], 0, m_digits[13]});
    *this = s1 + s2 + s2 + s3 + s3 + s4 + s5 - s6 - s7 - s8 - s9;
}
\end{cppcode}
Умные дяди написали, что нужно именно так, поэтому мы как макаки повторяем. Значит умножение можно сделать через:
\begin{cppcode}
friend constexpr F_256 operator*(const F_256& lhs, const F_256& rhs) {
    F_256 result;

    for (size_t i = 0; i < c_half_digit_number; ++i) {
        uint64_t u = 0;

        for (size_t j = 0; j < c_half_digit_number; ++j) {
            u = static_cast<uint64_t>(result[i + j])
              + static_cast<uint64_t>(lhs[i]) * static_cast<uint64_t>(rhs[j]) + (u >> c_digit_size);
            result[i + j] = static_cast<uint32_t>(u);
        }

        result[i + c_half_digit_number] = static_cast<uint32_t>(u >> c_digit_size);
    }

    result.reduce();
    return result;
}
\end{cppcode}
где используется только половина длины числа у i и j, так как мы считаем, что lhs и rhs удовлетворяют инваринату: lhs, rhs < p. 
\subsubsection{Инверсия}
Так как мы хотим и рыбку съесть и ..., то у нас нет деления. Но как же реализовать инверсию без деления? На помощь приходит алгоритм бинарного расширенного евклида. В его основе лежит житейская мудрость: не можешь использовать обычную арифметику --- используй бинарную. Так и поступил автор данного алгоритма. Пруфов работы не будет (точнее, объяснение есть в \cite{das2004guide}):
\begin{cppcode}
constexpr void inverse() {
    constexpr F_256 zero;
    constexpr F_256 one = 1;

    F_256 u = *this;
    F_256 v(p_values);

    F_256 x_1 = one;
    F_256 x_2;

    while (u != one && v != one) {
        while (u.is_even()) {
            u >>= 1;

            if (x_1.is_even()) {
                x_1 >>= 1;
            } else {
                x_1.add_p_uncheck();
                x_1 >>= 1;
                assert(x_1.is_valid());
            }
        }

        while (v.is_even()) {
            v >>= 1;

            if (x_2.is_even()) {
                x_2 >>= 1;
            } else {
                x_2.add_p_uncheck();
                x_2 >>= 1;
                assert(x_2.is_valid());
            }
        }

        if (u >= v) {
            u -= v;
            x_1 -= x_2;
        } else {
            v -= u;
            x_2 -= x_1;
        }
    }

    if (u == 1) {
        *this = x_1;
    } else {
        *this = x_2;
    }

    assert(is_valid());
}
\end{cppcode}
\subsubsection{Пролом 4 стены}
Нам нужен быстрый доступ к данным, поэтому открываем пользователю методы:
\begin{cppcode}
const uint32_t& first_digit() const {
    return m_digits[0];
}

constexpr const uint32_t& operator[](size_t pos) const {
    return m_digits[pos];
}

constexpr uint32_t& operator[](size_t pos) {
    return m_digits[pos];
}
\end{cppcode}
Да, это не безопасно, зато быстро.
\subsubsection{Конверсия}
У нас нет деления, значит мы можем привести только к 16,8,2-ичным системам счисления. Ограничимся для нашей специализации 16-ричной системой.

Основная сложность - каждая цифра содержит 8 16-ричных символов, поэтому нам понадобятся некоторые лямбды, которые будут упрощать деление числа на хекс-символы.

Так как мы начинаем заполнять со старших битов, то нам нужно пропустить верхние нули. Для цифр это несложно, но вот для скипа нулей внутри цифры нам придётся передать флаг в лямбды.

Итого:
\begin{cppcode}
constexpr std::string convert_to_hex_string() const {
    std::string result;
    size_t pos = c_digit_number;

    while (pos > 0 && m_digits[pos - 1] == 0) {
        --pos;
    }

    if (pos == 0) {
        return "0x0";
    }

    result += "0x";

    auto mini_push = [&](const uint8_t& value) {
        if (value < 10) {
            result.push_back(value + '0');
        } else {
            result.push_back(value - 10 + 'A');
        }
    };

    auto push = [&](const uint32_t& value, bool first_time = false) {
        size_t shift = 32;
        uint8_t temp = 0;

        if (first_time) {
            while (shift > 0 && temp == 0) {
                shift -= 4;
                temp = (value >> shift) & 0xF;
            }

            do {
                mini_push(temp);

                if (shift == 0) {
                    break;
                }

                shift -= 4;
                temp = (value >> shift) & 0xF;
            } while (shift >= 0);

            return;
        }

        while (shift > 0) {
            shift -= 4;
            temp = (value >> shift) & 0xF;
            mini_push(temp);
        }
    };

    push(m_digits[pos - 1], true);
    --pos;

    while (pos > 0) {
        push(m_digits[pos - 1]);
        --pos;
    }

    return result;
}
\end{cppcode}
Остальные методы без изменений.
\subsection{Точка Эллиптической кривой}
Забываем о классе эллиптической кривой и захардкоживаем внутрь класса точки основные параметры:
\begin{cppcode}
class EllipticCurvePoint {
    static constexpr F_256 c_a = "0xffffffff00000001000000000000000000000000fffffffffffffffffffffffc";
    static constexpr F_256 c_b = "0x5ac635d8aa3a93e7b3ebbd55769886bc651d06b0cc53b0f63bce3c3e27d2604b";
    static constexpr F_256 c_p_1 = "0x7FFFFFFF800000008000000000000000000000007FFFFFFFFFFFFFFFFFFFFFFF";
    static constexpr F_256 c_p_2 = "0x3FFFFFFFC0000000400000000000000000000000400000000000000000000000";

    F_256 m_x;
    F_256 m_y;
    bool m_is_null;
};
\end{cppcode}
где $p_1$ и $p_2$ - те самые значения, которые мы считали при нахождении корня в поле:
\[p_1:=\frac{p-1}{2}\]
\[p_2:=\frac{p+1}{4}\]
Они понадобятся нам для имплементации приватного метода нахождения y-координаты. Заметим, что $p\equiv 3\mod 4$, поэтому можно использовать прежний алгоритм:
\begin{cppcode}
static constexpr std::optional<F_256> find_y(const F_256& x) {
    F_256 value = F_256::pow(x, 3) + c_a * x + c_b;

    if (!value.is_invertible()) {
        return std::nullopt;
    }

    if (F_256::pow(value, c_p_1) != F_256(1)) {
        return std::nullopt;
    }

    return F_256::pow(value, c_p_2);
}
\end{cppcode}
В остальном никаких изменений методов нет, мы просто копируем используемые алгоритмы, например wNAF, в наш файл и подставляем в нужные места.
\subsection{Эль-Гамаль}
Так как нам больше не нужно держать общие данные, то класс Эль-Гамаль распадается на обычный namespace:
\begin{cppcode}
namespace ElGamal {
    using Point = EllipticCurvePoint;

    namespace {
        static constexpr Point m_generator =
            Point(F_256("0x6b17d1f2e12c4247f8bce6e563a440f277037d812deb33a0f4a13945d898c296"),
                  F_256("0x4fe342e2fe1a7f9b8ee7eb4a7c0f9e162bce33576b315ececbb6406837bf51f5"));
        static constexpr F_256 m_generator_order =
            "0xffffffff00000000ffffffffffffffffbce6faada7179e84f3b9cac2fc632551";

        Point map_to_curve(const F_256& message) {
            for (;;) {
                F_256 x = generate_random_non_zero_value_modulo(m_generator_order);

                for (size_t i = 0; i < 6; ++i) {
                    x[i] = message[i];
                }

                auto opt = Point::point_with_x_equal_to(x);

                if (opt.has_value()) {
                    return opt.value();
                }
            }
        }

        F_256 map_to_uint(const Point& message) {
            F_256 result;
            const F_256& x = message.get_x();

            for (size_t i = 0; i < 6; ++i) {
                result[i] = x[i];
            }

            return result;
        }
    }   // namespace

    struct Keys {
        F_256 private_key;
        Point public_key;
    };

    struct EncryptedMessage {
        Point generator_degree;
        Point message_with_salt;
    };

    constexpr Keys generate_keys() {
        F_256 private_key = generate_random_non_zero_value_modulo(m_generator_order);
        Point public_key = private_key * m_generator;
        return Keys {.private_key = private_key, .public_key = public_key};
    }

    EncryptedMessage encrypt(const F_256& message, const Point& public_key) {
        const F_256 k = generate_random_non_zero_value_modulo(m_generator_order);
        const Point generator_degree = k * m_generator;
        const Point message_with_salt = map_to_curve(message) + k * public_key;
        return {.generator_degree = generator_degree, .message_with_salt = message_with_salt};
    }

    F_256 decrypt(const EncryptedMessage& encrypted_message, const F_256& private_key) {
        Point M = encrypted_message.message_with_salt - private_key * encrypted_message.generator_degree;
        return map_to_uint(M);
    }
};   // namespace ElGamal
\end{cppcode}
Имплементация функций остаётся прежней, кроме того факта, что мы используем 6 бит из 8, вместо 4 бит, во время отображения сообщения на кривую и обратно. Это позволяет значительно увеличить передаваемое сообщение.
\subsection{Тестирование}
Наконец мы можем побаловаться с шифрованием сообщений через консоль:
\begin{cppcode}
int main() {
    std::cout << "Enter hexadecimal message:\n";
    std::string msg;
    std::cin >> msg;
    F_256 message = msg.c_str();
    std::cout << "Generating keys...\n";
    auto keys = ElGamal::generate_keys();
    std::cout << "Encrypting message...\n";
    auto encrypted_message = ElGamal::encrypt(message, keys.public_key);
    std::cout << "Decrypting message...\n";
    auto decrypted_message = ElGamal::decrypt(encrypted_message, keys.private_key);
    std::cout << "Decrypted message is:\n";
    std::string decrypted_msg = decrypted_message.convert_to_hex_string();
    std::cout << decrypted_msg << '\n';
    std::cout << "Checking correctness...\n";

    if (message != decrypted_message) {
        std::cout << "Fail ";

        if (msg.ends_with(decrypted_msg.substr(2))) {
            std::cout << "due to insufficient number of encryption bits\n";
        } else {
            std::cout << "due to implementation problems\n";
        }
    } else {
        std::cout << "Success!\n";
    }
}
\end{cppcode}
Результаты замеров тестов, сравнивая с шаблонным многофайловым решением:
\begin{itemize}
  \item Debug сборка:

    TODO

  \item Release сборка:

    TODO
\end{itemize}

Вывод: TODO

