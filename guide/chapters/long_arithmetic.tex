\ProvidesFile{long_arithmetic.tex}

\section{Длинная арифметика}
\textbf{Если не хотите запариваться и сделать основную рабочую лошадку блэкбоксом, то можно просто установить длинку boost::multiprecision и скипнуть данную часть.}

Так как мы хотим реализовать вычеты по большому модулю, то достаточно реализовать беззнаковые длинные целые числа.

Основной приём для имплементации длинной арифметики - хранение чисел в основании $2^{32}$ или $2^{64}$. То есть просто массив из целых чисел, которые представляют цифры данного числа в соответствующих основаниях. Есть несколько видов данного представления:
\begin{enumerate}
  \item Количество цифр в числе меняется в зависимости от размера числа. Нет математического ограничения длины числа, только аппаратное.
  \item Количество цифр в числе фиксировано и не меняется от числа.
\end{enumerate}
Последний вариант можно видеть, например, в типе uint64\_t --- присутствуют сразу все 64 бита не зависимо от содержащихся данных. Данный вид целых чисел является наиболее удобным в реализации и использовании по назначению, поэтому будем имплементировать его.

\subsection{Каркас}
Так как количество бит в числе может разительно отличатся от задачи к задаче, то общим решением будет создать шаблонный класс по количеству содержащихся в нём бит:
\begin{cppcode}
template<size_t c_bits>
class uint_t {
};
\end{cppcode}
Теперь надо определиться с представлением цифр в нашем классе. Из-за того, что для алгоритма деления, который будет позже, потребуется деление по две цифры, то возьмём за цифру uint32\_t, чтобы можно было спокойно делить в uin64\_t.

С помощью constexpr определим размеры и длину необходимого массива. Так как длина не изменяется во время жизни объекта, то возмём std::array за контейнер. Он будет гарантировать, что длина массива сохраняет свой инвариант. Итого получилось:
\begin{cppcode}
template<size_t c_bits>
class uint_t {
    using digit_t = uint32_t;
    using double_digit_t = uint64_t;

    static constexpr size_t c_bits_in_byte = 8;
    static constexpr size_t c_digit_size = sizeof(digit_t) * c_bits_in_byte;
    static constexpr size_t c_digit_number = c_bits / c_digit_size;
    static constexpr size_t c_double_digit_size = sizeof(double_digit_t) * c_bits_in_byte;
    static constexpr size_t c_double_digit_number = c_bits / c_double_digit_size;

    template<size_t V>
    friend class uint_t;

    using digits = std::array<digit_t, c_digit_number>;
    digits m_digits = {};
};
\end{cppcode}
Дали псевдонимы используемым типам, чтобы улучшить читаемость и не менять все типы, если вдруг захотим использовать за цифру uint16\_t и какой-то другой тип. В $\rm m\_digits$ храним число в основании $2^{32}$ в little-endian. 12-13 строчкой мы подружили все шаблоны друг с другом для общего взаимодействия.
\subsection{Методы}
Теперь мы хотим как-то общаться с нашими данными. Желательно имплементировать все операции, которые можно применять к обычным типам, таким как uint32\_t, uint64\_t. 
\begin{itemize}
  \item Конструирование:
    Есть несколько сценариев:
    \begin{enumerate}
      \item Хотим сконструироваться от целочисленного типа. Заметим, что количество бит в нём может быть как больше 32, так и меньше, поэтому ограничиться назначением $\rm m\_blocks[0]$ нельзя. Тогда определим вспомогательный приватный метод, который будет определять через концепты отношения количества бит:
      \begin{cppcode}
template<typename From, typename To>
concept is_upcastable_to = sizeof(From) <= sizeof(To) && is_convertible_to<From, To>;

template<typename From, typename To>
concept is_downcastable_to = sizeof(From) > sizeof(To) && is_convertible_to<From, To>;

template<typename T>
requires std::numeric_limits<T>::is_integer && is_upcastable_to<T, digit_t>
static constexpr digits split_into_digits(T value) {
    return {static_cast<digit_t>(value)};
}

template<typename T>
requires std::numeric_limits<T>::is_integer && is_downcastable_to<T, digit_t>
static constexpr digits split_into_digits(T value) {
    digits result = {};

    for (size_t i = 0; i < c_digit_number; ++i) {
        result[i] = static_cast<digit_t>(value);
        value >>= c_digit_size;

        if (value == 0) {
            break;
        }
    }

    return result;
}
      \end{cppcode}
      \item Хотим сконструироваться от контейнера целых чисел, например как наш uint\_t. Для этого пишем концепт, который определяет, что данный тип действительно является контейнером целых чисел, и копируем его данные:
      \begin{cppcode}
template<typename Container, typename T>
concept is_convertible_container = requires(Container t, size_t i) {
    { t[i] } -> is_convertible_to<T>;
    { t.size() } -> std::same_as<size_t>;
};
        template<typename T>
requires is_convertible_container<T, digit_t> || requires(T x) {
    { uint_t {x} } -> std::same_as<T>;
}
static constexpr digits split_into_digits(const T& other) {
    const size_t min_size = std::min(size(), other.size());
    digits result = {};

    for (size_t i = 0; i < min_size; i++) {
        result[i] = static_cast<digit_t>(other[i]);
    }

    return result;
}
      \end{cppcode}
      \item Хотим сконструироваться от строк (C-строк). Действительно, это единственный удобный способ задать необходимое нам число с количеством бит больше чем у size\_t.

      Для этого нам нужно сначала написать парсер строки, которая представляет число в двоичном, шестнадцатиричном и десяточном форматах. Так как мы хотим использовать методы uint\_t, то нам нужно парсить в любой класс, который удовлетворяет критериям целочисленных классов:
      \begin{cppcode}
template<typename T>
concept is_integral = std::is_integral_v<T> || requires(T t, T* p, void (*f)(T)) {
    f(0);
    p + t;
};
template<typename T>
requires is_integral<T>
constexpr T parse_into_uint(const char* str) {
    assert(str != nullptr && "parse_into got nullptr");

    T value = 0;
    uint16_t radix = 10;

    if (str[0] == '0' && str[1] == 'x') {
        radix = 16;
        str += 2;
    } else if (str[0] == '0' && str[1] == 'b') {
        radix = 2;
        str += 2;
    } else if (str[0] == '0') {
        radix = 8;
        ++str;
    }

    while (*str != '\0') {
        value *= static_cast<T>(radix);
        uint16_t symbol_value = radix + 1;

        if (*str >= '0' && *str <= '9') {
            symbol_value = static_cast<uint16_t>(*str - '0');
        } else if (*str >= 'a' && *str <= 'f') {
            symbol_value = static_cast<uint16_t>(*str - 'a') + 10;
        } else if (*str >= 'A' && *str <= 'F') {
            symbol_value = static_cast<uint16_t>(*str - 'A') + 10;
        }

        if (symbol_value >= radix) {
            assert(false && "parse_into got incorrect string");
        }

        value += static_cast<T>(symbol_value);
        ++str;
    }

    return value;
}
      \end{cppcode}
    \end{enumerate}
    Теперь мы готовы определить конструкторы класса:
    \begin{cppcode}
constexpr uint_t() = default;

template<typename T>
constexpr uint_t(const T& value) : m_digits(split_into_digits<T>(value)) {}

constexpr uint_t(const char* str) : m_digits(parse_into_uint<uint_t>(str).m_digits) {};

constexpr uint_t& operator=(const uint_t& value) = default;
    \end{cppcode}
    Специально не делаем их explicit для неявных конвертаций. С помощью шаблонного конструктора мы можем конструироваться от других инстансов нашего класса, например:
    \begin{cppcode}
uint_t<128> a = ...;
uint_t<160> b(a);
    \end{cppcode}
  \item Сложение:
    Самое простое, но тем не менее лучшее решение - это сложение в столбик. Используем стандартизированное переполнение беззнаковых типов в C++ для определения, есть ли остаток от сложения наших 32-битных чисел:
    \begin{cppcode}
constexpr uint_t& operator+=(const uint_t& other) {
    digit_t carry = 0;

    for (size_t i = 0; i < c_digit_number; ++i) {
        digit_t sum = carry + other[i];
        m_digits[i] += sum;
        carry = (m_digits[i] < sum) || (sum < carry);
    }

    return *this;
}
    \end{cppcode}
    Используем ключевое слово constexpr для вычисления значения некоторых констант во время компиляции.

    Теперь мы хотим определить простое сложение, т.е. operator+. Его можно было бы сделать через
    \begin{cppcode}
constexpr uint_t operator+(const uint_t& other) const {
    uint_t result = *this;
    result += other;
    return result;
}
    \end{cppcode}
    Обратите внимание, что мы не пишем:
    \begin{cppcode}
constexpr uint_t operator+(const uint_t& other) const {
    uint_t result = *this;
    return result += other;
}
    \end{cppcode}
    так как тогда мы будем возвращать ссылку uint\_t\&, что не затриггерит $\rm NRVO$.

    Вместо определения метода класса, мы напишем 4 дружественных функции:
    \begin{cppcode}
friend constexpr uint_t operator+(const uint_t& lhs, const uint_t& rhs) {
    uint_t result = lhs;
    result += rhs;
    return result;
}

friend constexpr uint_t operator+(uint_t&& lhs, const uint_t& rhs) {
    lhs += rhs;
    return lhs;
}

friend constexpr uint_t operator+(const uint_t& lhs, uint_t&& rhs) {
    rhs += lhs;
    return rhs;
}

friend constexpr uint_t operator+(uint_t&& lhs, uint_t&& rhs) {
    lhs += rhs;
    return lhs;
}
    \end{cppcode}
    Тут есть два оптимизационных момента:
    \begin{enumerate}
      \item Эффективно используется то, что нам передали $\rm r-value$, и не копируем данные. Обычно это возникает при многократном сложении или в других сложных формулах:
      \begin{cppcode}
uint_t a,b,c = ...;
uint_t result = a + b + c;
      \end{cppcode}
      \item Теперь можно неявно заапкастить другие типы к uint\_t, чтобы применить данное сложение. Это позволяет писать:
      \begin{cppcode}
uint_t a = ...;
uint_t b = 3 + a;
      \end{cppcode}
      что было бы невозможно при внутреннем методе. Везде далее будем использовать по возможности внешние friend функции для возможности неявного апкаста других типов.
    \end{enumerate}
  \item Вычитание:
  Оно абсолютно аналогично делается через вычитание в столбик:
  \begin{cppcode}
constexpr uint_t& operator-=(const uint_t& other) {
    digit_t remainder = 0;

    for (size_t i = 0; i < c_digit_number; ++i) {
        digit_t prev = m_digits[i];
        digit_t sum = other[i] + remainder;
        m_digits[i] -= sum;
        remainder = (m_digits[i] > prev) || (sum < remainder);
    }

    return *this;
}
  \end{cppcode}
  Так как мы хотим писать такие конструкции как:
  \begin{cppcode}
uint_t a = ...;
uint_t b = -a;
  \end{cppcode}
  то нужно определить отрицание. В компьютерах отрицательные целые числа представляются как флипнутые биты + 1. Рассмотрим на примере:
  \[000000000000000000101\]
  \[+\]
  \[111111111111111111011\]
  \[=\]
  \[000000000000000000000\]
  Определим inplace отрицание как приватный метод:
  \begin{cppcode}
constexpr void negative() {
    for (size_t i = 0; i < c_digit_number; ++i) {
        m_digits[i] = ~(m_digits[i]);
    }

    ++*this;
}
  \end{cppcode}
  Значит отрицанием будет:
  \begin{cppcode}
constexpr uint_t operator-() const {
    uint_t result = *this;
    result.negative();
    return result;
}
  \end{cppcode}
  Определяем внешние friend для вычитания:
  \begin{cppcode}
friend constexpr uint_t operator-(const uint_t& lhs, const uint_t& rhs) {
    uint_t result = lhs;
    result -= rhs;
    return result;
}

friend constexpr uint_t operator-(uint_t&& lhs, const uint_t& rhs) {
    lhs -= rhs;
    return lhs;
}

friend constexpr uint_t operator-(const uint_t& lhs, uint_t&& rhs) {
    rhs -= lhs;
    rhs.negative();
    return rhs;
}

friend constexpr uint_t operator-(uint_t&& lhs, uint_t&& rhs) {
    lhs -= rhs;
    return lhs;
}
  \end{cppcode}
  \item Умножение:
    Самым лёгким и относительно быстрым способом будет метод сканирования операнда (или построчное умножение). Заключается в построчном обновлении цифр ответа и накоплении прежних результатов умножения для новой строки. Получается такое динамическое программирование:
    \begin{cppcode}
friend constexpr uint_t operator*(const uint_t& lhs, const uint_t& rhs) {
    uint_t result;

    for (size_t i = 0; i < c_digit_number; ++i) {
        uint64_t u = 0;

        for (size_t j = 0; i + j < c_digit_number; ++j) {
            u = static_cast<uint64_t>(result[i + j])
              + static_cast<uint64_t>(lhs[i]) * static_cast<uint64_t>(rhs[j])
              + (u >> c_digit_size);
            result[i + j] = static_cast<uint32_t>(u);
        }
    }

    return result;
}
    \end{cppcode}
    Заметим, что мы отбрасываем цифры, чей номер не помещается в c\_digit\_number. Улучшения этого алгоритма можно найти в \cite{fast_mult}.

    Также в реализации имплементировано \href{https://neerc.ifmo.ru/trains/toulouse/2017/fft2.pdf}{нерекурсивное FFT} умножение двух комплексных полиномов, но из-за большой ошибки при конвертации обратно в целочисленные цифры он выдаёт неверный ответ уже при длине числа в две цифры, что непозволительно для умножения. Поэтому он не используется и не рассматривается в этом гайде. Если вы смогли реализовать более точный FFT или NTT, то можете попробовать использовать их.
  \item Деление:
    Чтобы поделить два целочисленных длинных числа используем алгоритм-D Кнута \cite{knuth2014art}:

    Задача - поделить два длинных числа, представленных цифрами с основанием $b$, где $b$ в имплементации $2^{32}$ или $2^{64}$. 

   Рассмотрим сначала $u=(u_nu_{n-1}\dots,u_0)_b$ и $v=(v_{n-1}\dots v_0)_b$, где $u/v <b$. Найдём алгоритм для вычисления $q:=\lfloor u/v \rfloor$:

   Заметим, что $u/v<b\Leftrightarrow u/b < v\Leftrightarrow \lfloor u/b\rfloor < v$, а это условие того, что \[(u_nu_{n-1}\dots,u_1)_b <  (v_{n-1}\dots v_0)_b\]
   Если обозначить $r:=u - qv$, то $q$ - это уникальное число, такое что $0\leqslant r <v$. Пусть
   \[\hat{q}:=\min \left(\left\lfloor \frac{u_nb + u_{n-1}}{v_{n-1}}\right\rfloor, b-1\right)\]
   Т.е. мы получаем гипотетическое значение $q$, поделив первые две цифры $u$ на первую цифру $v$, а если результат деления больше или равен $b$, то берём $b-1$. Для такого $\hat{q}$ выполняются две теоремы: 
   \begin{theorem}
    $\hat{q}\geqslant q$
   \end{theorem}
   \begin{theorem}
     Если $v_{n-1}\geqslant \lfloor b/2\rfloor$, то $\hat{q}-2\leqslant q\leqslant \hat{q}$.
   \end{theorem}
   Существенно ограничили нашу гипотезу. Умножив $u$ и $v$ на $\lfloor b/(v_{n-1}+1)\rfloor$, мы не изменим длину числа $v$ и результат деления. После этого умножения станет выполнятся вторая из данных теорем.

   \paragraph{Алгоритм $D$:} Дано неотрицательное целое число $u = (u_{m+n-1},\dots,u_1,u_0)_b$ и $v = (v_{n-1},\dots,v_1,v_0)_b$, где $v_{n-1}\neq 0$ и $n>1$. Мы хотим посчитать $\lfloor u/v\rfloor = (q_m,q_{m-1},\dots,q_0)_b$ и остаток $u$ mod $v$ $= (r_{n-1},\dots,r_0)_b$:

   \begin{enumerate}
     \item $d:=\lfloor b/(v_{n-1}+1)\rfloor$. Тогда пусть $(u_{m+n}u_{m+n-1}\dots u_1u_0)_b:=(u_{m+n-1}\dots u_1u_0)_b\cdot d$, аналогично, $(v_{n-1},\dots,v_1,v_0)_b =(v_{n-1},\dots,v_1,v_0)_b\cdot d$. Заметим, что новая цифра могла появиться только у $u$
     \item Итерироваться будем по $j$, которая в начале равна $m$ (Делить в следующих шагах будем $(u_{j+n}\dots u_{j+1}u_j)_b$ на $(v_{n-1}\dots v_1v_0)_b$ чтобы получить цифру $q_j$) 
     \item $\hat{q} := \left\lfloor \frac{u_{j+n}b + u_{j+n-1}}{v_{n-1}}\right\rfloor$ и пусть $\hat{r}$ будет остатком, т.е. $\hat{r} :=u_{j+n}b + u_{j+n-1}$ (mod $v_{n-1}$)
     \item Если $\hat{q}\geqslant b$ или $\hat{q}v_{n-2}>b\hat{r} + u_{j+n-2}$, то уменьшаем $\hat{q}$ на 1 и увеличиваем $\hat{r}$ на $v_{n-1}$. Если $\hat{r}<b$, то повторяем данный шаг
     \item Заменим $(u_{j+n}\dots u_{j+1}u_j)_b$ на
     \[(u_{j+n}\dots u_{j+1}u_j)_b - \hat{q}(0v_{n-1}\dots v_1v_0)_b\]
     \item Назначаем $q_j = \hat{q}$
     \item Если число $u$ на 5 шаге получилось отрицательным, то добавляем к нему $b^{n+1}$ и переходим к шагу 8, иначе переходим к шагу 9.
     \item (Вероятность данного шага крайне мала, за счёт чего достигается асимптотическая быстрота алгоритма) Уменьшаем $q_j$ на 1 и добавляем $(0v_{n-1}\dots v_1v_0)_b$ к $(u_{j+n}\dots u_{j+1}u_j)_b$ (при сложении появится цифра $u_{j+n+1}$, её следует проигнорировать)
     \item Уменьшаем $j$ на 1. Если $j\geqslant 0$, то возвращаемся на шаг 3
     \item Теперь $q=(q_m\dots q_1q_0)$ - это искомое частное, а искомый остаток можно получить, поделив $(u_{n-1}\dots u_1u_0)$ на $d$ наивным способом 
     \item Возвращаем $(q,r)$
   \end{enumerate}
   Алгоритм сверху применяется только при размере делителя больше 1 цифры и не больше количества цифр в делимом, так как при меньших размерах есть более быстрые оптимизации, значит нам понадобится приватный метод определения количества цифр в числе:
   \begin{cppcode}
constexpr size_t actual_size() const {
    size_t result = c_digit_number;

    while (result > 0 && m_digits[result - 1] == 0) {
        --result;
    }

    return result;
}
   \end{cppcode}
   Для удобной работы с $\rm m\_digits$ определим приватные методы для operator[], которые будут проталкивать его внутрь:
   \begin{cppcode}
constexpr const digit_t& operator[](size_t pos) const {
    return m_digits[pos];
}

constexpr digit_t& operator[](size_t pos) {
    return m_digits[pos];
}
   \end{cppcode}
   Наконец, определим приватный метод divide, который будет вычислять, в каком случае мы находимся.
   \begin{cppcode}
static constexpr uint_t divide(const uint_t& lhs, const uint_t& rhs, uint_t* remainder = nullptr) {
    size_t dividend_size = lhs.actual_size();
    size_t divisor_size = rhs.actual_size();

    // CASE 0:
    if (dividend_size < divisor_size) {
        if (remainder != nullptr) {
            *remainder = lhs;
        }

        return uint_t(0);
    }

    // CASE 1:
    if (divisor_size == 1) {
        return divide(lhs, rhs[0], remainder);
    }

    // CASE 2:
    return d_divide(lhs, rhs, remainder);
}

static constexpr uint_t divide(const uint_t& lhs, const digit_t& rhs, uint_t* remainder = nullptr) {
    uint_t result;
    double_digit_t part = 0;

    for (size_t i = c_digit_number; i > 0; --i) {
        part = (part << (c_digit_size)) + static_cast<double_digit_t>(lhs[i - 1]);

        if (part < rhs) {
            continue;
        }

        result[i - 1] = static_cast<digit_t>(part / rhs);
        part %= rhs;
    }

    if (remainder != nullptr) {
        *remainder = uint_t(static_cast<digit_t>(part));
    }

    return result;
}

static constexpr uint_t d_divide(const uint_t& lhs, const uint_t& rhs, uint_t* remainder = nullptr) {
    size_t dividend_size = lhs.actual_size();
    size_t divisor_size = rhs.actual_size();

    uint_t<c_bits + c_digit_size> dividend(lhs);
    uint_t divisor(rhs);
    uint_t quotient;

    size_t shift_size = 0;
    digit_t divisor_head = divisor[divisor_size - 1];
    static constexpr double_digit_t c_HalfBlock = static_cast<double_digit_t>(1)
                                               << (c_digit_size - 1);

    while (divisor_head < c_HalfBlock) {
        ++shift_size;
        divisor_head <<= 1;
    }

    dividend <<= shift_size;
    divisor <<= shift_size;

    double_digit_t divisor_ = divisor[divisor_size - 1];
    static constexpr double_digit_t c_Block = static_cast<double_digit_t>(1) << c_digit_size;

    for (size_t i = dividend_size - divisor_size + 1; i > 0; --i) {
        double_digit_t part =
            (static_cast<double_digit_t>(dividend[i + divisor_size - 1]) << c_digit_size)
            + static_cast<double_digit_t>(dividend[i + divisor_size - 2]);
        double_digit_t quotient_temp = part / divisor_;
        part %= divisor_;

        if (quotient_temp == c_Block) {
            --quotient_temp;
            part += divisor_;
        }

        while (part < c_Block
               && (quotient_temp * divisor[divisor_size - 2]
                   > (part << c_digit_size) + dividend[i + divisor_size - 3])) {
            --quotient_temp;
            part += divisor_;
        }

        int64_t carry = 0;
        int64_t widedigit = 0;

        for (size_t j = 0; j < divisor_size; ++j) {
            double_digit_t product =
                static_cast<digit_t>(quotient_temp) * static_cast<double_digit_t>(divisor[j]);
            widedigit = (static_cast<int64_t>(dividend[i + j - 1]) + carry) - (product & UINT32_MAX);
            dividend[i + j - 1] = static_cast<digit_t>(widedigit);
            carry = (widedigit >> c_digit_size) - static_cast<double_digit_t>(product >> c_digit_size);
        }

        widedigit = static_cast<int64_t>(dividend[i + divisor_size - 1]) + carry;
        dividend[i + divisor_size - 1] = static_cast<digit_t>(widedigit);

        quotient[i - 1] = static_cast<digit_t>(quotient_temp);

        if (widedigit < 0) {
            --quotient[i - 1];
            widedigit = 0;

            for (size_t j = 0; j < divisor_size; ++j) {
                widedigit += static_cast<double_digit_t>(dividend[i + j - 1]) + divisor[j];
                dividend[i + j - 1] = static_cast<digit_t>(widedigit);
                widedigit >>= 32;
            }
        }
    }

    if (remainder != nullptr) {
        *remainder = uint_t(0);

        for (size_t i = 0; i < divisor_size - 1; ++i) {
            (*remainder)[i] =
                (dividend[i] >> shift_size)
                | (static_cast<double_digit_t>(dividend[i + 1]) << (c_digit_size - shift_size));
        }

        (*remainder)[divisor_size - 1] = dividend[divisor_size - 1] >> shift_size;
    }

    return quotient;
}
   \end{cppcode}
   Теперь можем определить операторы деления и остатка:
   \begin{cppcode}
friend constexpr uint_t operator/(const uint_t& lhs, const uint_t& rhs) {
    uint_t result = divide(lhs, rhs);
    uint_t less = result * rhs;
    uint_t greater = (result + 1) * rhs;
    if (less > lhs || greater <= lhs) {
        result = 0;
    }
    return result;
}

friend constexpr uint_t operator%(const uint_t& lhs, const uint_t& rhs) {
    uint_t remainder;
    divide(lhs, rhs, &remainder);
    return remainder;
}

constexpr uint_t& operator*=(const uint_t& other) {
    return *this = *this * other;
}

constexpr uint_t& operator/=(const uint_t& other) {
    return *this = *this / other;
}
   \end{cppcode}
   Так как нам не нужны r-value при делении, то не пишем оптимизации на них.
  \item Битовые сдвиги:
  Нам поступает запрос на сдвиг на $\rm size\_t \hspace{0.1cm} shift$ бит влево или вправо. Для высокой производительности выполнение операции нужно разбить на два этапа: сдвиг цифр внутри числа, сдвиг битов внутри цифр:
  \begin{cppcode}
constexpr uint_t& operator>>=(size_t shift_size) {
    size_t digit_shift = shift_size >> 5;

    if (digit_shift > 0) {
        for (size_t i = 0; i < c_digit_number; ++i) {
            if (i + digit_shift < c_digit_number) {
                m_digits[i] = m_digits[i + digit_shift];
            } else {
                m_digits[i] = 0;
            }
        }
    }

    shift_size %= c_digit_size;

    if (shift_size == 0) {
        return *this;
    }

    for (size_t i = 0; i + digit_shift < c_digit_number; ++i) {
        m_digits[i] >>= shift_size;

        if (i + 1 < c_digit_number) {
            m_digits[i] |= m_digits[i + 1] << (c_digit_size - shift_size);
        }
    }

    return *this;
}

constexpr uint_t& operator<<=(size_t shift_size) {
    size_t digit_shift = shift_size >> 5;

    if (digit_shift > 0) {
        for (size_t i = c_digit_number; i > 0; --i) {
            if (i > digit_shift) {
                m_digits[i - 1] = m_digits[i - digit_shift - 1];
            } else {
                m_digits[i - 1] = 0;
            }
        }
    }

    shift_size %= c_digit_size;

    if (shift_size == 0) {
        return *this;
    }

    for (size_t i = c_digit_number; i > digit_shift; --i) {
        m_digits[i - 1] <<= shift_size;

        if (i - 1 > 0) {
            m_digits[i - 1] |= m_digits[i - 2] >> (c_digit_size - shift_size);
        }
    }

    return *this;
}
  \end{cppcode}
  В обоих методах в концах двигаем недостающие биты из соседней цифры, если она существует. Определяем внешние friend:
  \begin{cppcode}
friend constexpr uint_t operator>>(const uint_t& lhs, const size_t& rhs) {
    uint_t result = lhs;
    return result >>= rhs;
}

friend constexpr uint_t operator>>(uint_t&& lhs, const size_t& rhs) {
    return lhs >>= rhs;
}

friend constexpr uint_t operator<<(const uint_t& lhs, const size_t& rhs) {
    uint_t result = lhs;
    return result <<= rhs;
}

friend constexpr uint_t operator<<(uint_t&& lhs, const size_t& rhs) {
    return lhs <<= rhs;
}
  \end{cppcode}
  \item Сравнение:
  Так как используются 20 плюсы, то можно определить оператор <=>, но мы не можем использовать $\rm = default$, так как тогда сравнение будет с 0 индекса, а не с последнего:
    \begin{cppcode}
friend constexpr std::strong_ordering operator<=>(const uint_t& lhs, const uint_t& rhs) {
    for (size_t i = c_digit_number; i > 0; --i) {
        if (lhs[i - 1] != rhs[i - 1]) {
            return lhs[i - 1] <=> rhs[i - 1];
        }
    }

    return std::strong_ordering::equal;
}
    \end{cppcode}
    Так как мы не определили через default, нам придётся написать и оператор равенства, но он очевиден:
    \begin{cppcode}
friend constexpr bool operator==(const uint_t& lhs, const uint_t& rhs) {
    return lhs.m_digits == rhs.m_digits;
}
    \end{cppcode}
  \item Битовые операции:
  Наше число является по сути большой последовательностью бит одного числа, поэтому битовые операции выполняются поэлементно:
  \begin{cppcode}
constexpr uint_t& operator^=(const uint_t& other) {
    for (size_t i = 0; i < c_digit_number; ++i) {
        m_digits[i] ^= other[i];
    }

    return *this;
}

constexpr uint_t& operator|=(const uint_t& other) {
    for (size_t i = 0; i < c_digit_number; ++i) {
        m_digits[i] |= other[i];
    }

    return *this;
}

constexpr uint_t& operator&=(const uint_t& other) {
    for (size_t i = 0; i < c_digit_number; ++i) {
        m_digits[i] &= other[i];
    }

    return *this;
}
  \end{cppcode}
  Определяем внешние friend:
  \begin{cppcode}
friend constexpr uint_t operator^(const uint_t& lhs, const uint_t& rhs) {
    uint_t result = lhs;
    return result ^= rhs;
}

friend constexpr uint_t operator^(uint_t&& lhs, const uint_t& rhs) {
    return lhs ^= rhs;
}

friend constexpr uint_t operator^(const uint_t& lhs, uint_t&& rhs) {
    return rhs ^= lhs;
}

friend constexpr uint_t operator^(uint_t&& lhs, uint_t&& rhs) {
    return lhs ^= rhs;
}

friend constexpr uint_t operator|(const uint_t& lhs, const uint_t& rhs) {
    uint_t result = lhs;
    result |= rhs;
    return result;
}

friend constexpr uint_t operator|(uint_t&& lhs, const uint_t& rhs) {
    lhs |= rhs;
    return lhs;
}

friend constexpr uint_t operator|(const uint_t& lhs, uint_t&& rhs) {
    rhs |= lhs;
    return rhs;
}

friend constexpr uint_t operator|(uint_t&& lhs, uint_t&& rhs) {
    lhs |= rhs;
    return lhs;
}

friend constexpr uint_t operator&(const uint_t& lhs, const uint_t& rhs) {
    uint_t result = lhs;
    result &= rhs;
    return result;
}

friend constexpr uint_t operator&(uint_t&& lhs, const uint_t& rhs) {
    lhs &= rhs;
    return lhs;
}

friend constexpr uint_t operator&(const uint_t& lhs, uint_t&& rhs) {
    rhs &= lhs;
    return rhs;
}

friend constexpr uint_t operator&(uint_t&& lhs, uint_t&& rhs) {
    lhs &= rhs;
    return lhs;
}
  \end{cppcode}
  \item Унарные инкремент и декремент:
  Определим вспомогательные приватные методы для увеличения/уменьшения числа на 1, скопировав код у $+=/-=$ соответственно:
  \begin{cppcode}
constexpr void increment() {
    for (size_t i = 0; i < c_digit_number; ++i) {
        m_digits[i] += 1;

        if (m_digits[i] != 0) {
            break;
        }
    }
}

constexpr void decrement() {
    for (size_t i = 0; i < c_digit_number; ++i) {
        digit_t temp = m_digits[i];
        m_digits[i] -= 1;

        if (temp >= m_digits[i]) {
            break;
        }
    }
}
  \end{cppcode}
  То есть мы пытаемся прибавить/вычесть остаток, пока не найдём хотя бы одну цифру, которая не переполниться от этой операции.

  Теперь сами унарные операции. В C++ они бывают двух видов: префиксные и постфиксные. Они разделяются типом int в аргументе:
  \begin{cppcode}
[[nodiscard("Optimize unary operator usage")]]
constexpr uint_t
    operator++(int) {
    uint_t result = *this;
    increment();
    return result;
}

constexpr uint_t& operator++() {
    increment();
    return *this;
}

[[nodiscard("Optimize unary operator usage")]]
constexpr uint_t
    operator--(int) {
    uint_t result = *this;
    decrement();
    return result;
}

constexpr uint_t& operator--() {
    decrement();
    return *this;
}
  \end{cppcode}
  Дали атрибуты nodiscard постфиксным операторам, чтобы пользователь эффиктивно использовал унарны операции.
  \item Конвертация в стандартные типы:
  Два варианта:
  \begin{enumerate}
    \item Хотим обрезать наш тип до стандартных целочисленных типов. Тогда заполняем требуемый тип битами из m\_digits:
    \begin{cppcode}
template<typename T>
requires is_convertible_to<T, digit_t>
constexpr T convert_to() const {
    size_t shift_size = sizeof(T) * c_bits_in_byte;
    size_t digits_number = shift_size / c_digit_size;

    if (digits_number == 0) {
        return static_cast<T>(m_digits[0]);
    }

    T result = 0;

    for (size_t i = 0; i < c_digit_number && i < digits_number; ++i) {
        result |= static_cast<T>(m_digits[i]) << (i * c_digit_size);
    }

    return result;
} 
    \end{cppcode}
    \item Хотим получить всё число. Так как ни один стандартный тип такого размера не поддерживает, то будем переводить наше число в строку обычным делением:
    \begin{cppcode}
template<typename T>
constexpr T convert_to() const;

template<>
constexpr std::string convert_to() const {
    std::string result;
    uint_t clone_of_this = *this;

    do {
        uint_t remainder;
        clone_of_this = divide(clone_of_this, 10, &remainder);
        result.push_back(remainder.m_digits[0] + '0');
    } while (clone_of_this > 0);

    std::reverse(result.begin(), result.end());
    return result;
}
    \end{cppcode}
  \end{enumerate}
  Заметим, что мы специально сделали шаблон, а потом его специализировали, чтобы можно было использовать два варианта одинаково:
    \begin{cppcode}
uint_t a = ...;
size_t b = a.convert_to<size_t>();
std::string s = a.convert_to<std::string>();
    \end{cppcode}
\end{itemize}

\subsection{Главный тип}
Независимо от того, выбрали мы писать свою длинку или использовать boost::multiprecision, нам нужно задать главный целочисленный тип, на котором мы будем в дальнейшем работать.

Так как обычно для алгоритмов шифрования на эллиптических кривых мы будем использовать эллиптические кривые, \href{https://nvlpubs.nist.gov/nistpubs/SpecialPublications/NIST.SP.800-186-draft.pdf}{предложенные NIST}, то нам нужно выбрать количество бит, которое в два раза больше количетсва бит используемых чисел, чтобы числа не переполнялись при умножении. В данном гайде будем ориентироваться на \href{https://neuromancer.sk/std/nist/P-256}{NIST P-256}, поэтому нам понадобится 512 бит в нашей длинке.

Два варианта:
\begin{enumerate}
  \item Для бустовской длинки: 
  \begin{cppcode}
#include "boost/multiprecision/cpp_int.hpp"
#include "boost/multiprecision/fwd.hpp"

using uint = boost::multiprecision::uint512_t;
  \end{cppcode}
  \item Для самописной:
  \begin{cppcode}
#include "long-arithmetic.h"

using uint = uint_t<512>;
  \end{cppcode}
\end{enumerate}

